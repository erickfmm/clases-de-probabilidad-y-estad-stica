\documentclass[aspectratio=169]{beamer}
\usetheme{Madrid}
\usecolortheme{default}
\usepackage[utf8]{inputenc}
\usepackage[spanish]{babel}
\usepackage{amsmath}
\usepackage{amssymb}
\usepackage{tikz}
\usepackage{booktabs}

% Configuración del tema
\setbeamertemplate{navigation symbols}{}
\setbeamertemplate{footline}[frame number]

% Comandos personalizados
\newcommand{\alert}[1]{\textcolor{red}{#1}}
\newcommand{\highlight}[1]{\textcolor{blue}{#1}}

% Variables del documento (serán reemplazadas por Jinja2)
\title{Introducción a la Probabilidad y Estadística}
\subtitle{¡Descubriendo el mundo de los números y las posibilidades!}
\author{Probabilidad y Estadística}
\date{\today}

\begin{document}

% Página de título
\begin{frame}
\titlepage
\end{frame}

% Diapositivas generadas dinámicamente
\begin{frame}{¿Qué vamos a aprender?}
\begin{alertblock}{Nota}
¡Bienvenidos a una aventura numérica! 🎲📊
\end{alertblock}
\begin{itemize}
\item Dos mundos fascinantes:
\end{itemize}
\begin{itemize}
\item 📊 Estadística: entender lo que ya pasó
\end{itemize}
\begin{itemize}
\item 🎲 Probabilidad: predecir lo que puede pasar
\end{itemize}
\begin{block}{Ejemplo}
Como un detective que analiza pistas (estadística) y adivina quién es el culpable (probabilidad)
\end{block}
\end{frame}

\begin{frame}{¿Qué es la Estadística?}
\begin{itemize}
\item Es el arte de recolectar, organizar y analizar datos
\end{itemize}
\begin{itemize}
\item Nos ayuda a entender el pasado y el presente
\end{itemize}
\begin{alertblock}{Nota}
¡Es como ser un detective de números! 🔍
\end{alertblock}
\begin{block}{Ejemplo}
Contar cuántos estudiantes prefieren pizza vs hamburguesa
\end{block}
\end{frame}

\begin{frame}{La Estadística en tu Vida Diaria}
\begin{itemize}
\item 📱 Redes sociales: conteo de likes y seguidores
\end{itemize}
\begin{itemize}
\item ⚽ Deportes: estadísticas de goles y victorias
\end{itemize}
\begin{itemize}
\item 🎮 Videojuegos: puntuaciones y rankings
\end{itemize}
\begin{itemize}
\item 📺 Netflix: series más vistas
\end{itemize}
\begin{alertblock}{Nota}
¡Estás rodeado de estadísticas sin darte cuenta!
\end{alertblock}
\end{frame}

\begin{frame}{Ejemplo Divertido de Estadística}
\begin{exampleblock}{Problema}
Preguntamos a 10 amigos su sabor de helado favorito
\end{exampleblock}
\begin{center}
\begin{tabular}{|c|c|)}
\toprule
Sabor & Votos \\
\midrule
Chocolate & 4 \\
Vainilla & 3 \\
Fresa & 2 \\
Limón & 1 \\
\bottomrule
\end{tabular}
\end{center}
\begin{align*}
Conclusión: ¡El chocolate es el campeón! 🍫
\end{align*}
\end{frame}

\begin{frame}{¿Qué es la Probabilidad?}
\begin{itemize}
\item Es calcular qué tan posible es que algo ocurra
\end{itemize}
\begin{itemize}
\item Trabaja con eventos FUTUROS o inciertos
\end{itemize}
\begin{alertblock}{Nota}
¡Es como predecir el futuro con matemáticas! 🔮
\end{alertblock}
\begin{block}{Ejemplo}
¿Qué probabilidad hay de que llueva mañana?
\end{block}
\end{frame}

\begin{frame}{La Probabilidad en tu Vida}
\begin{itemize}
\item 🎲 Juegos de mesa: lanzar dados
\end{itemize}
\begin{itemize}
\item 🃏 Cartas: ¿me saldrá un as?
\end{itemize}
\begin{itemize}
\item ⚽ Deportes: ¿ganará mi equipo?
\end{itemize}
\begin{itemize}
\item 🌦️ Clima: pronóstico del tiempo
\end{itemize}
\begin{itemize}
\item 🎰 Sorteos y rifas
\end{itemize}
\begin{alertblock}{Nota}
¡La probabilidad está en cada decisión!
\end{alertblock}
\end{frame}

\begin{frame}{Ejemplo Divertido de Probabilidad}
\begin{exampleblock}{Problema}
Tienes una bolsa con 10 caramelos: 6 rojos y 4 azules
\end{exampleblock}
\begin{itemize}
\item Sin mirar, sacas uno. ¿Qué color es más probable?
\end{itemize}
\begin{align*}
Rojo: 6 de cada 10 (60%)
\end{align*}
\begin{align*}
Azul: 4 de cada 10 (40%)
\end{align*}
\begin{itemize}
\item ¡Es más probable sacar uno rojo! 🔴
\end{itemize}
\end{frame}

\begin{frame}{Diferencias Clave}
\begin{center}
\begin{tabular}{|c|c|)}
\toprule
Estadística & Probabilidad \\
\midrule
Analiza datos del PASADO & Predice el FUTURO \\
¿Qué pasó? & ¿Qué puede pasar? \\
Certeza & Incertidumbre \\
Temperaturas del mes & Pronóstico del tiempo \\
\bottomrule
\end{tabular}
\end{center}
\begin{alertblock}{Nota}
¡Son dos caras de la misma moneda! 🪙
\end{alertblock}
\end{frame}

\begin{frame}{¿Por qué son Importantes?}
\begin{itemize}
\item 🏥 Medicina: estudios de efectividad de tratamientos
\end{itemize}
\begin{itemize}
\item 💼 Negocios: decidir qué productos vender
\end{itemize}
\begin{itemize}
\item 🎓 Educación: mejorar métodos de enseñanza
\end{itemize}
\begin{itemize}
\item 🌍 Ciencia: descubrir patrones en la naturaleza
\end{itemize}
\begin{alertblock}{Nota}
¡Ayudan a tomar mejores decisiones!
\end{alertblock}
\end{frame}

\begin{frame}{Vocabulario Básico}
\begin{itemize}
\item 📊 Datos: información que recolectamos
\end{itemize}
\begin{itemize}
\item 👥 Población: todos los elementos que estudiamos
\end{itemize}
\begin{itemize}
\item 🎯 Muestra: parte de la población
\end{itemize}
\begin{itemize}
\item 🎲 Evento: algo que puede ocurrir
\end{itemize}
\begin{itemize}
\item 🎯 Experimento: acción que produce resultados
\end{itemize}
\begin{block}{Ejemplo}
Experimento: lanzar una moneda. Evento: sale cara
\end{block}
\end{frame}

\begin{frame}{Un Experimento Divertido}
\begin{exampleblock}{Problema}
Lanza una moneda 10 veces y anota los resultados
\end{exampleblock}
\begin{itemize}
\item Estadística: cuenta cuántas caras y cuántas sellos
\end{itemize}
\begin{itemize}
\item Probabilidad: ¿cuál es la chance teórica?
\end{itemize}
\begin{align*}
Teóricamente: 50% cara, 50% sello
\end{align*}
\begin{alertblock}{Nota}
¿Coincide tu experimento con la teoría?
\end{alertblock}
\end{frame}

\begin{frame}{Escalera de la Probabilidad}
\begin{itemize}
\item Imposible (0%): que salga un 7 en un dado normal 🎲
\end{itemize}
\begin{itemize}
\item Poco probable (25%): sacar una carta de corazones ♥️
\end{itemize}
\begin{itemize}
\item Igual de probable (50%): que salga cara 🪙
\end{itemize}
\begin{itemize}
\item Muy probable (75%): NO sacar un 6 en un dado
\end{itemize}
\begin{itemize}
\item Seguro (100%): que el sol salga mañana ☀️
\end{itemize}
\begin{alertblock}{Nota}
¡Todo evento tiene su lugar en la escalera!
\end{alertblock}
\end{frame}

\begin{frame}{Actividad Práctica}
\begin{exampleblock}{Problema}
Piensa en estas situaciones y clasifícalas:
\end{exampleblock}
\begin{itemize}
\item 1. ¿Es estadística o probabilidad?
\end{itemize}
\begin{itemize}
\item 2. Calificaciones del año pasado
\end{itemize}
\begin{itemize}
\item 3. Probabilidad de ganar un juego
\end{itemize}
\begin{itemize}
\item 4. Número de goles en el mundial anterior
\end{itemize}
\begin{itemize}
\item 5. Chance de que nieve en tu ciudad
\end{itemize}
\end{frame}

\begin{frame}{¿Listo para la Aventura?}
\begin{itemize}
\item ✅ Ya sabes qué es la estadística
\end{itemize}
\begin{itemize}
\item ✅ Ya sabes qué es la probabilidad
\end{itemize}
\begin{itemize}
\item ✅ Viste ejemplos de la vida real
\end{itemize}
\begin{alertblock}{Nota}
¡Ahora estás listo para profundizar en cada tema!
\end{alertblock}
\begin{itemize}
\item En las próximas clases aprenderemos:
\end{itemize}
\begin{itemize}
\item 📊 Tablas y gráficos
\end{itemize}
\begin{itemize}
\item 📏 Medidas estadísticas
\end{itemize}
\begin{itemize}
\item 🎲 Cálculos de probabilidad
\end{itemize}
\end{frame}


% Diapositiva final
\begin{frame}
\begin{center}
\Huge ¿Preguntas?
\end{center}
\end{frame}

\end{document}