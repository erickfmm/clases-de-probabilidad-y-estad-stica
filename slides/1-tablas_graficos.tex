\documentclass[aspectratio=169]{beamer}
\usetheme{Madrid}
\usecolortheme{default}
\usepackage[utf8]{inputenc}
\usepackage[spanish]{babel}
\usepackage{amsmath}
\usepackage{amssymb}
\usepackage{tikz}
\usepackage{booktabs}

% Configuración del tema
\setbeamertemplate{navigation symbols}{}
\setbeamertemplate{footline}[frame number]

% Comandos personalizados
\newcommand{\alert}[1]{\textcolor{red}{#1}}
\newcommand{\highlight}[1]{\textcolor{blue}{#1}}

% Variables del documento (serán reemplazadas por Jinja2)
\title{Representación de datos a través de tablas y gráficos}
\subtitle{¡Organicemos los datos de forma divertida! 📊}
\author{Probabilidad y Estadística}
\date{\today}

\begin{document}

% Página de título
\begin{frame}
\titlepage
\end{frame}

% Diapositivas generadas dinámicamente
\begin{frame}{¿Por qué organizar datos?}
\begin{alertblock}{Nota}
¡Imagina tu cuarto desordenado vs ordenado! 🏠
\end{alertblock}
\begin{itemize}
\item Datos desordenados = confusión 😵
\end{itemize}
\begin{itemize}
\item Datos organizados = claridad y respuestas ✨
\end{itemize}
\begin{block}{Ejemplo}
¿Qué helado prefieren tus amigos? Con una tabla lo sabrás al instante
\end{block}
\end{frame}

\begin{frame}{¿Qué son los datos?}
\begin{itemize}
\item Son información que recolectamos
\end{itemize}
\begin{itemize}
\item Pueden ser números: edades, pesos, calificaciones
\end{itemize}
\begin{itemize}
\item O categorías: colores, sabores, deportes
\end{itemize}
\begin{block}{Ejemplo}
Datos de tu clase: alturas de todos los estudiantes
\end{block}
\begin{alertblock}{Nota}
¡Los datos cuentan historias, solo hay que saber leerlos! 📖
\end{alertblock}
\end{frame}

\begin{frame}{Tablas de Frecuencia - ¿Qué son?}
\begin{itemize}
\item Es como hacer un conteo organizado 📝
\end{itemize}
\begin{itemize}
\item Agrupamos datos iguales y contamos
\end{itemize}
\begin{alertblock}{Nota}
¡Es como ordenar tu colección de cartas por tipo!
\end{alertblock}
\begin{block}{Ejemplo}
Si tienes muchas manzanas rojas y pocas verdes, la tabla lo muestra claramente
\end{block}
\end{frame}

\begin{frame}{Frecuencia Absoluta}
\begin{itemize}
\item Es simplemente: ¿cuántas veces aparece?
\end{itemize}
\begin{itemize}
\item Es el conteo directo, el número total
\end{itemize}
\begin{alertblock}{Nota}
¡Es como contar con los dedos! ✋
\end{alertblock}
\begin{block}{Ejemplo}
Colores de autos en el estacionamiento:
\end{block}
\begin{itemize}
\item Rojos: 5, Azules: 3, Blancos: 7
\end{itemize}
\begin{itemize}
\item Las frecuencias absolutas son: 5, 3 y 7
\end{itemize}
\end{frame}

\begin{frame}{Frecuencia Relativa}
\begin{itemize}
\item Es la proporción del total
\end{itemize}
\begin{itemize}
\item Responde: ¿qué parte del todo representa?
\end{itemize}
\begin{center}
\Large
$\text{Frec. Relativa} = \frac{\text{Frec. Absoluta}}{\text{Total}}$
\end{center}
\begin{alertblock}{Nota}
¡Es como saber qué porción de pizza te toca! 🍕
\end{alertblock}
\begin{block}{Ejemplo}
Si hay 10 autos y 5 son rojos: 5/10 = 0.5 = 50%
\end{block}
\end{frame}

\begin{frame}{Ejemplo Paso a Paso}
\begin{exampleblock}{Problema}
Notas de 6 estudiantes: 15, 18, 15, 20, 18, 15
\end{exampleblock}
\begin{itemize}
\item Paso 1: Identificar valores diferentes: 15, 18, 20
\end{itemize}
\begin{itemize}
\item Paso 2: Contar cada uno (Frec. Absoluta)
\end{itemize}
\begin{itemize}
\item Paso 3: Calcular proporción (Frec. Relativa)
\end{itemize}
\begin{alertblock}{Nota}
¡Vamos a construir la tabla juntos!
\end{alertblock}
\end{frame}

\begin{frame}{Tabla Completa del Ejemplo}
\begin{center}
\begin{tabular}{|c|c|c|)}
\toprule
Nota & Frec. Abs. & Frec. Rel. \\
\midrule
15 & 3 & 3/6 = 0.50 \\
18 & 2 & 2/6 = 0.33 \\
20 & 1 & 1/6 = 0.17 \\
\bottomrule
\end{tabular}
\end{center}
\begin{itemize}
\item Total de estudiantes: 6
\end{itemize}
\begin{align*}
¡La nota 15 es la más frecuente! Aparece en la mitad de casos
\end{align*}
\begin{alertblock}{Nota}
Nota: todas las frec. relativas suman 1.00 (o 100%)
\end{alertblock}
\end{frame}

\begin{frame}{Gráfico de Barras 📊}
\begin{itemize}
\item Barras verticales u horizontales
\end{itemize}
\begin{itemize}
\item Cada barra = una categoría
\end{itemize}
\begin{itemize}
\item Altura de barra = frecuencia
\end{itemize}
\begin{alertblock}{Nota}
¡Perfecto para comparar categorías de un vistazo!
\end{alertblock}
\begin{block}{Ejemplo}
Deportes favoritos: Fútbol (10), Básquet (7), Vóley (5)
\end{block}
\begin{itemize}
\item ¡Las barras muestran claramente cuál es más popular! ⚽
\end{itemize}
\end{frame}

\begin{frame}{Histograma 📈}
\begin{itemize}
\item Similar a barras, pero para datos numéricos continuos
\end{itemize}
\begin{itemize}
\item Agrupa datos en rangos o intervalos
\end{itemize}
\begin{itemize}
\item Las barras están pegadas (sin espacios)
\end{itemize}
\begin{block}{Ejemplo}
Edades: 10-15, 15-20, 20-25 años
\end{block}
\begin{alertblock}{Nota}
¡Muestra cómo se distribuyen los datos!
\end{alertblock}
\end{frame}

\begin{frame}{Gráfico Circular (Pastel) 🥧}
\begin{itemize}
\item Un círculo dividido en porciones
\end{itemize}
\begin{itemize}
\item Cada porción = una categoría
\end{itemize}
\begin{itemize}
\item Tamaño de porción = frecuencia relativa
\end{itemize}
\begin{alertblock}{Nota}
¡Ideal para ver proporciones del 100%!
\end{alertblock}
\begin{block}{Ejemplo}
Presupuesto familiar: 50% comida, 30% vivienda, 20% otros
\end{block}
\end{frame}

\begin{frame}{Gráfico de Líneas 📉}
\begin{itemize}
\item Puntos conectados con líneas
\end{itemize}
\begin{itemize}
\item Perfecto para mostrar cambios en el tiempo
\end{itemize}
\begin{block}{Ejemplo}
Temperatura de la semana, ventas mensuales
\end{block}
\begin{alertblock}{Nota}
¡Puedes ver tendencias: ¿sube? ¿baja? ¿se mantiene?
\end{alertblock}
\begin{itemize}
\item Las líneas te cuentan la historia del cambio 📖
\end{itemize}
\end{frame}

\begin{frame}{¿Cuál gráfico usar?}
\begin{center}
\begin{tabular}{|c|c|)}
\toprule
Tipo de datos & Gráfico ideal \\
\midrule
Categorías & Barras o Circular \\
Números continuos & Histograma \\
Cambios en tiempo & Líneas \\
\bottomrule
\end{tabular}
\end{center}
\begin{alertblock}{Nota}
¡Elige según lo que quieras mostrar! 🎯
\end{alertblock}
\end{frame}

\begin{frame}{El Promedio - ¿Qué es?}
\begin{itemize}
\item También llamado 'media aritmética'
\end{itemize}
\begin{itemize}
\item Es el valor central, el punto de balance ⚖️
\end{itemize}
\begin{alertblock}{Nota}
¡Imagina repartir todo en partes iguales!
\end{alertblock}
\begin{block}{Ejemplo}
Si 3 amigos tienen 3, 6 y 9 caramelos, el promedio es repartir equitativamente
\end{block}
\end{frame}

\begin{frame}{¿Cómo calcular el Promedio?}
\begin{itemize}
\item Paso 1: Suma todos los valores
\end{itemize}
\begin{itemize}
\item Paso 2: Divide entre cuántos valores hay
\end{itemize}
\begin{center}
\Large
$\bar{x} = \frac{\text{suma de todos}}{\text{cantidad de datos}}$
\end{center}
\begin{alertblock}{Nota}
¡Es como compartir todo y ver cuánto le toca a cada uno! 🍰
\end{alertblock}
\end{frame}

\begin{frame}{Ejemplo del Promedio}
\begin{block}{Ejemplo}
Datos: 5, 8, 10, 12, 15
\end{block}
\begin{itemize}
\item Paso 1: Sumar todos
\end{itemize}
\begin{align*}
5 + 8 + 10 + 12 + 15 = 50
\end{align*}
\begin{itemize}
\item Paso 2: Dividir entre cantidad (5 datos)
\end{itemize}
\begin{align*}
\bar{x} = \frac{50}{5} = 10
\end{align*}
\begin{alertblock}{Nota}
¡El promedio es 10! Es el valor 'típico' del grupo
\end{alertblock}
\end{frame}

\begin{frame}{Practicando con Frutas 🍎}
\begin{exampleblock}{Problema}
Manzanas vendidas por día: 12, 15, 10, 18, 20
\end{exampleblock}
\begin{itemize}
\item ¿Cuántas manzanas se venden en promedio por día?
\end{itemize}
\begin{align*}
Suma: 12 + 15 + 10 + 18 + 20 = 75
\end{align*}
\begin{align*}
Promedio: 75 ÷ 5 = 15 manzanas/día
\end{align*}
\begin{itemize}
\item ¡En promedio se venden 15 manzanas diarias! 🎯
\end{itemize}
\end{frame}

\begin{frame}{Ejercicio Divertido 🌡️}
\begin{exampleblock}{Problema}
Temperaturas de la semana (°C): 22, 24, 21, 23, 25, 20, 22
\end{exampleblock}
\begin{itemize}
\item Tarea 1: Construir tabla de frecuencias
\end{itemize}
\begin{itemize}
\item Tarea 2: Calcular temperatura promedio
\end{itemize}
\begin{itemize}
\item Tarea 3: ¿Qué gráfico usarías?
\end{itemize}
\begin{alertblock}{Nota}
¡Piensa qué historia quieres contar con los datos!
\end{alertblock}
\end{frame}

\begin{frame}{Solución - Tabla}
\begin{center}
\begin{tabular}{|c|c|)}
\toprule
Temperatura & Días (Frec. Abs.) \\
\midrule
20°C & 1 \\
21°C & 1 \\
22°C & 2 \\
23°C & 1 \\
24°C & 1 \\
25°C & 1 \\
\bottomrule
\end{tabular}
\end{center}
\begin{alertblock}{Nota}
22°C es la más frecuente (aparece 2 veces)
\end{alertblock}
\end{frame}

\begin{frame}{Solución - Promedio}
\begin{align*}
Suma: 22+24+21+23+25+20+22 = 157
\end{align*}
\begin{align*}
Promedio: 157 ÷ 7 = 22.43°C
\end{align*}
\begin{itemize}
\item Interpretación:
\end{itemize}
\begin{itemize}
\item La temperatura típica de la semana fue 22.4°C
\end{itemize}
\begin{alertblock}{Nota}
¡Es una semana cálida y estable! ☀️
\end{alertblock}
\end{frame}

\begin{frame}{Solución - Gráfico}
\begin{itemize}
\item Mejor opción: Gráfico de líneas 📈
\end{itemize}
\begin{itemize}
\item ¿Por qué?
\end{itemize}
\begin{itemize}
\item ✓ Muestra cambios día a día
\end{itemize}
\begin{itemize}
\item ✓ Podemos ver si subió o bajó
\end{itemize}
\begin{itemize}
\item ✓ Identifica tendencias
\end{itemize}
\begin{alertblock}{Nota}
También podríamos usar barras para comparar cada día
\end{alertblock}
\end{frame}

\begin{frame}{¡Resumen de la clase!}
\begin{itemize}
\item ✅ Tablas: organizan datos claramente
\end{itemize}
\begin{itemize}
\item ✅ Frecuencia absoluta: conteo directo
\end{itemize}
\begin{itemize}
\item ✅ Frecuencia relativa: proporción del total
\end{itemize}
\begin{itemize}
\item ✅ Gráficos: visualizan datos
\end{itemize}
\begin{itemize}
\item ✅ Promedio: valor típico del grupo
\end{itemize}
\begin{alertblock}{Nota}
¡Ahora puedes organizar y entender cualquier conjunto de datos! 🎉
\end{alertblock}
\end{frame}


% Diapositiva final
\begin{frame}
\begin{center}
\Huge ¿Preguntas?
\end{center}
\end{frame}

\end{document}