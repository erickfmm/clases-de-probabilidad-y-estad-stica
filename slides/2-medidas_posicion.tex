\documentclass[aspectratio=169]{beamer}
\usetheme{Madrid}
\usecolortheme{default}
\usepackage[utf8]{inputenc}
\usepackage[spanish]{babel}
\usepackage{amsmath}
\usepackage{amssymb}
\usepackage{tikz}
\usepackage{booktabs}

% Configuración del tema
\setbeamertemplate{navigation symbols}{}
\setbeamertemplate{footline}[frame number]

% Comandos personalizados
\newcommand{\alert}[1]{\textcolor{red}{#1}}
\newcommand{\highlight}[1]{\textcolor{blue}{#1}}

% Variables del documento (serán reemplazadas por Jinja2)
\title{Medidas de Posición}
\subtitle{¡Descubriendo dónde están los datos! 📍}
\author{Probabilidad y Estadística}
\date{\today}

\begin{document}

% Página de título
\begin{frame}
\titlepage
\end{frame}

% Diapositivas generadas dinámicamente
\begin{frame}{¿Qué son las Medidas de Posición?}
\begin{alertblock}{Nota}
¡Imagina una fila de personas ordenadas por altura! 👥
\end{alertblock}
\begin{itemize}
\item Nos ayudan a ubicar datos dentro de un grupo
\end{itemize}
\begin{itemize}
\item Responden preguntas como:
\end{itemize}
\begin{itemize}
\item ¿Dónde está el dato del medio?
\end{itemize}
\begin{itemize}
\item ¿Cuál es el valor que supera el 75% de los datos?
\end{itemize}
\begin{block}{Ejemplo}
Como saber en qué puesto quedaste en una carrera 🏃
\end{block}
\end{frame}

\begin{frame}{La Mediana - El Valor del Centro}
\begin{itemize}
\item Es el dato que está justo en el medio
\end{itemize}
\begin{itemize}
\item 50% de datos están abajo, 50% arriba
\end{itemize}
\begin{alertblock}{Nota}
¡Es como el estudiante del medio en una fila ordenada!
\end{alertblock}
\begin{block}{Ejemplo}
Alturas: 1.50, 1.60, 1.70, 1.80, 1.90 m
\end{block}
\begin{align*}
Mediana = 1.70 m (el del medio)
\end{align*}
\end{frame}

\begin{frame}{Encontrar la Mediana}
\begin{itemize}
\item Paso 1: Ordena los datos de menor a mayor
\end{itemize}
\begin{itemize}
\item Paso 2: Si hay cantidad impar, toma el del centro
\end{itemize}
\begin{itemize}
\item Paso 3: Si hay cantidad par, promedia los 2 del centro
\end{itemize}
\begin{block}{Ejemplo}
Datos: 3, 7, 2, 9, 5 → Ordenar: 2, 3, 5, 7, 9
\end{block}
\begin{align*}
Mediana = 5 (posición central)
\end{align*}
\end{frame}

\begin{frame}{Mediana con Cantidad Par}
\begin{block}{Ejemplo}
Datos: 4, 8, 6, 2, 10, 12
\end{block}
\begin{itemize}
\item Ordenar: 2, 4, 6, 8, 10, 12
\end{itemize}
\begin{itemize}
\item Los dos centrales son: 6 y 8
\end{itemize}
\begin{align*}
Mediana = (6 + 8) ÷ 2 = 7
\end{align*}
\begin{alertblock}{Nota}
¡La mediana no siempre es un dato original!
\end{alertblock}
\end{frame}

\begin{frame}{¿Qué son los Cuartiles?}
\begin{itemize}
\item Dividen los datos en 4 partes iguales
\end{itemize}
\begin{alertblock}{Nota}
¡Como cortar una pizza en 4 partes! 🍕
\end{alertblock}
\begin{itemize}
\item Q1: primer cuartil (25%)
\end{itemize}
\begin{itemize}
\item Q2: segundo cuartil = mediana (50%)
\end{itemize}
\begin{itemize}
\item Q3: tercer cuartil (75%)
\end{itemize}
\begin{block}{Ejemplo}
Separan grupos de 25%, 50%, 75% y 100%
\end{block}
\end{frame}

\begin{frame}{Cuartiles Explicados}
\begin{itemize}
\item Q1: 25% de datos están por debajo
\end{itemize}
\begin{itemize}
\item Q2: 50% de datos están por debajo (mediana)
\end{itemize}
\begin{itemize}
\item Q3: 75% de datos están por debajo
\end{itemize}
\begin{alertblock}{Nota}
¡Son como marcas en una regla! 📏
\end{alertblock}
\begin{block}{Ejemplo}
Si Q3 de salarios es 3000, significa que 75% gana menos de 3000
\end{block}
\end{frame}

\begin{frame}{Ejemplo de Cuartiles Paso a Paso}
\begin{exampleblock}{Problema}
Datos: 2, 4, 6, 8, 10, 12, 14, 16, 18
\end{exampleblock}
\begin{itemize}
\item Ya están ordenados (9 valores)
\end{itemize}
\begin{align*}
Q2 (mediana) = 10 (valor central)
\end{align*}
\begin{itemize}
\item Mitad inferior: 2, 4, 6, 8
\end{itemize}
\begin{align*}
Q1 = (4+6)÷2 = 5
\end{align*}
\begin{itemize}
\item Mitad superior: 12, 14, 16, 18
\end{itemize}
\begin{align*}
Q3 = (14+16)÷2 = 15
\end{align*}
\end{frame}

\begin{frame}{Interpretando los Cuartiles}
\begin{block}{Ejemplo}
Q1=5, Q2=10, Q3=15
\end{block}
\begin{itemize}
\item Interpretación:
\end{itemize}
\begin{itemize}
\item 📊 25% de datos ≤ 5
\end{itemize}
\begin{itemize}
\item 📊 50% de datos ≤ 10
\end{itemize}
\begin{itemize}
\item 📊 75% de datos ≤ 15
\end{itemize}
\begin{alertblock}{Nota}
¡Los cuartiles dividen los datos en grupos iguales!
\end{alertblock}
\end{frame}

\begin{frame}{¿Qué son los Percentiles?}
\begin{itemize}
\item Dividen los datos en 100 partes iguales
\end{itemize}
\begin{alertblock}{Nota}
¡Como dividir una cinta métrica en 100 cm! 📐
\end{alertblock}
\begin{itemize}
\item P₁, P₂, P₃, ..., P₉₉
\end{itemize}
\begin{itemize}
\item Pₖ significa: k% de datos están por debajo
\end{itemize}
\begin{block}{Ejemplo}
P₉₀ en una prueba: superaste al 90% de estudiantes 🎓
\end{block}
\end{frame}

\begin{frame}{Relación entre Cuartiles y Percentiles}
\begin{center}
\begin{tabular}{|c|c|)}
\toprule
Cuartil & Percentil Equivalente \\
\midrule
Q1 & P₂₅ \\
Q2 (mediana) & P₅₀ \\
Q3 & P₇₅ \\
\bottomrule
\end{tabular}
\end{center}
\begin{alertblock}{Nota}
¡Los cuartiles son percentiles especiales!
\end{alertblock}
\begin{itemize}
\item P₁₀ = 10% por debajo, P₉₀ = 90% por debajo
\end{itemize}
\end{frame}

\begin{frame}{Ejemplo de Percentiles}
\begin{block}{Ejemplo}
En un examen obtuviste 85 puntos
\end{block}
\begin{itemize}
\item El P₉₀ es 80 puntos
\end{itemize}
\begin{itemize}
\item Interpretación:
\end{itemize}
\begin{itemize}
\item ✅ Superaste al 90% de estudiantes
\end{itemize}
\begin{itemize}
\item ✅ Solo el 10% sacó más que tú
\end{itemize}
\begin{alertblock}{Nota}
¡Excelente resultado! Estás en el top 10% 🌟
\end{alertblock}
\end{frame}

\begin{frame}{El Diagrama de Cajón (Box Plot)}
\begin{itemize}
\item Herramienta visual para ver distribución
\end{itemize}
\begin{alertblock}{Nota}
¡Es como un resumen gráfico de 5 números! 📦
\end{alertblock}
\begin{itemize}
\item Muestra de un vistazo:
\end{itemize}
\begin{itemize}
\item 🔹 El mínimo
\end{itemize}
\begin{itemize}
\item 🔹 Q1, Q2 (mediana), Q3
\end{itemize}
\begin{itemize}
\item 🔹 El máximo
\end{itemize}
\end{frame}

\begin{frame}{Partes del Diagrama de Cajón}
\begin{itemize}
\item 🔵 Caja: va de Q1 a Q3
\end{itemize}
\begin{itemize}
\item 📏 Rango intercuartílico (RIC) = Q3 - Q1
\end{itemize}
\begin{itemize}
\item ━ Línea en la caja: la mediana (Q2)
\end{itemize}
\begin{itemize}
\item 🔺 Bigotes: van desde la caja hasta mín y máx
\end{itemize}
\begin{alertblock}{Nota}
¡La caja contiene el 50% central de los datos!
\end{alertblock}
\end{frame}

\begin{frame}{Dibujando un Box Plot}
\begin{exampleblock}{Problema}
Datos: 5, 7, 8, 9, 10, 12, 15, 18, 20
\end{exampleblock}
\begin{align*}
Mínimo = 5, Máximo = 20
\end{align*}
\begin{align*}
Q1 = 8, Q2 = 10, Q3 = 15
\end{align*}
\begin{itemize}
\item Bigote izquierdo: de 5 a 8
\end{itemize}
\begin{itemize}
\item Caja: de 8 a 15 (con línea en 10)
\end{itemize}
\begin{itemize}
\item Bigote derecho: de 15 a 20
\end{itemize}
\end{frame}

\begin{frame}{Valores Atípicos (Outliers)}
\begin{itemize}
\item Datos muy alejados del resto
\end{itemize}
\begin{alertblock}{Nota}
¡Como alguien muy alto en una clase de niños pequeños! 🦒
\end{alertblock}
\begin{itemize}
\item Se detectan con la regla:
\end{itemize}
\begin{center}
\Large
$RIC = Q3 - Q1$
\end{center}
\begin{itemize}
\item Atípico si es < Q1 - 1.5×RIC
\end{itemize}
\begin{itemize}
\item o si es > Q3 + 1.5×RIC
\end{itemize}
\end{frame}

\begin{frame}{Ejemplo de Valor Atípico}
\begin{block}{Ejemplo}
Datos: 10, 12, 11, 13, 12, 50
\end{block}
\begin{align*}
Q1=11, Q3=13, RIC=13-11=2
\end{align*}
\begin{align*}
Límite superior: 13 + 1.5×2 = 16
\end{align*}
\begin{itemize}
\item El valor 50 > 16
\end{itemize}
\begin{alertblock}{Nota}
¡50 es un valor atípico! Se marca con un punto especial ⚫
\end{alertblock}
\end{frame}

\begin{frame}{Aplicación Práctica - Salarios}
\begin{exampleblock}{Problema}
Salarios mensuales (S/.): 1200, 1500, 1800, 2000, 2200, 2500, 3000, 3500, 5000
\end{exampleblock}
\begin{itemize}
\item Paso 1: Ya están ordenados ✓
\end{itemize}
\begin{itemize}
\item Paso 2: Calcular cuartiles
\end{itemize}
\begin{itemize}
\item Paso 3: Hacer diagrama de cajón
\end{itemize}
\begin{itemize}
\item Paso 4: Identificar atípicos
\end{itemize}
\end{frame}

\begin{frame}{Solución - Cuartiles}
\begin{align*}
Mínimo = 1200, Máximo = 5000
\end{align*}
\begin{align*}
Q2 (mediana) = 2200 (valor central)
\end{align*}
\begin{itemize}
\item Mitad baja: 1200, 1500, 1800, 2000
\end{itemize}
\begin{align*}
Q1 = (1500+1800)÷2 = 1650
\end{align*}
\begin{itemize}
\item Mitad alta: 2500, 3000, 3500, 5000
\end{itemize}
\begin{align*}
Q3 = (3000+3500)÷2 = 3250
\end{align*}
\end{frame}

\begin{frame}{Solución - Valores Atípicos}
\begin{align*}
RIC = 3250 - 1650 = 1600
\end{align*}
\begin{align*}
Límite inferior: 1650 - 1.5×1600 = -750
\end{align*}
\begin{align*}
Límite superior: 3250 + 1.5×1600 = 5650
\end{align*}
\begin{itemize}
\item Todos los salarios están entre -750 y 5650
\end{itemize}
\begin{alertblock}{Nota}
¡No hay valores atípicos! Todos son normales
\end{alertblock}
\end{frame}

\begin{frame}{Interpretación de Salarios}
\begin{itemize}
\item 📊 50% de trabajadores gana entre 1650 y 3250
\end{itemize}
\begin{itemize}
\item 📊 El salario típico (mediano) es 2200
\end{itemize}
\begin{itemize}
\item 📊 25% gana menos de 1650
\end{itemize}
\begin{itemize}
\item 📊 25% gana más de 3250
\end{itemize}
\begin{alertblock}{Nota}
¡El box plot muestra que hay variabilidad en los salarios!
\end{alertblock}
\end{frame}

\begin{frame}{Comparando con Box Plots}
\begin{itemize}
\item Podemos comparar dos o más grupos
\end{itemize}
\begin{block}{Ejemplo}
Comparar salarios de dos empresas lado a lado
\end{block}
\begin{itemize}
\item Se puede ver de un vistazo:
\end{itemize}
\begin{itemize}
\item ✓ Cuál tiene salarios más altos
\end{itemize}
\begin{itemize}
\item ✓ Cuál tiene más variabilidad
\end{itemize}
\begin{itemize}
\item ✓ Dónde hay valores atípicos
\end{itemize}
\begin{alertblock}{Nota}
¡Es súper útil para comparaciones! 👥
\end{alertblock}
\end{frame}

\begin{frame}{Ejercicio de Práctica}
\begin{exampleblock}{Problema}
Edades de un grupo: 15, 16, 15, 17, 16, 18, 15, 20, 16, 25
\end{exampleblock}
\begin{itemize}
\item Tarea 1: Ordenar los datos
\end{itemize}
\begin{itemize}
\item Tarea 2: Calcular Q1, Q2, Q3
\end{itemize}
\begin{itemize}
\item Tarea 3: Calcular RIC
\end{itemize}
\begin{itemize}
\item Tarea 4: Verificar si 25 es atípico
\end{itemize}
\begin{alertblock}{Nota}
¡Intenta resolverlo antes de ver la solución!
\end{alertblock}
\end{frame}

\begin{frame}{Solución del Ejercicio}
\begin{itemize}
\item Ordenado: 15, 15, 15, 16, 16, 16, 17, 18, 20, 25
\end{itemize}
\begin{align*}
Q1 = 15, Q2 = 16, Q3 = 18
\end{align*}
\begin{align*}
RIC = 18 - 15 = 3
\end{align*}
\begin{align*}
Límite superior: 18 + 1.5×3 = 22.5
\end{align*}
\begin{itemize}
\item 25 > 22.5
\end{itemize}
\begin{alertblock}{Nota}
¡Sí! 25 es un valor atípico (persona mayor en el grupo) ⚫
\end{alertblock}
\end{frame}

\begin{frame}{¡Resumen de la Clase!}
\begin{itemize}
\item ✅ Mediana: valor del centro (50%)
\end{itemize}
\begin{itemize}
\item ✅ Cuartiles: dividen en 4 partes (Q1, Q2, Q3)
\end{itemize}
\begin{itemize}
\item ✅ Percentiles: dividen en 100 partes
\end{itemize}
\begin{itemize}
\item ✅ Box Plot: visualiza la distribución
\end{itemize}
\begin{itemize}
\item ✅ Valores atípicos: datos muy alejados
\end{itemize}
\begin{alertblock}{Nota}
¡Ahora puedes analizar dónde se ubican los datos! 🎯
\end{alertblock}
\end{frame}


% Diapositiva final
\begin{frame}
\begin{center}
\Huge ¿Preguntas?
\end{center}
\end{frame}

\end{document}