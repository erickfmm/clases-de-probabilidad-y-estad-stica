\documentclass[aspectratio=169]{beamer}
\usetheme{Madrid}
\usecolortheme{default}
\usepackage[utf8]{inputenc}
\usepackage[spanish]{babel}
\usepackage{amsmath}
\usepackage{amssymb}
\usepackage{tikz}
\usepackage{booktabs}

% Configuración del tema
\setbeamertemplate{navigation symbols}{}
\setbeamertemplate{footline}[frame number]

% Comandos personalizados
\newcommand{\alert}[1]{\textcolor{red}{#1}}
\newcommand{\highlight}[1]{\textcolor{blue}{#1}}

% Variables del documento (serán reemplazadas por Jinja2)
\title{Reglas de las Probabilidades}
\subtitle{¡Calculando las posibilidades! 🎲}
\author{Probabilidad y Estadística}
\date{\today}

\begin{document}

% Página de título
\begin{frame}
\titlepage
\end{frame}

% Diapositivas generadas dinámicamente
\begin{frame}{¿Qué es la Probabilidad?}
\begin{itemize}
\item Es medir qué tan posible es que algo ocurra
\end{itemize}
\begin{alertblock}{Nota}
¡Es como adivinar con matemáticas! 🔮
\end{alertblock}
\begin{itemize}
\item Va de 0 (imposible) a 1 (seguro)
\end{itemize}
\begin{itemize}
\item También se expresa en porcentajes (0% a 100%)
\end{itemize}
\begin{block}{Ejemplo}
¿Qué probabilidad hay de que caiga cara al lanzar una moneda?
\end{block}
\end{frame}

\begin{frame}{La Escala de Probabilidad}
\begin{itemize}
\item 0 o 0%: Imposible (nunca pasa)
\end{itemize}
\begin{itemize}
\item 0.25 o 25%: Poco probable
\end{itemize}
\begin{itemize}
\item 0.5 o 50%: Igual de probable
\end{itemize}
\begin{itemize}
\item 0.75 o 75%: Muy probable
\end{itemize}
\begin{itemize}
\item 1 o 100%: Seguro (siempre pasa)
\end{itemize}
\begin{alertblock}{Nota}
¡Toda probabilidad está en esta escala! 📏
\end{alertblock}
\end{frame}

\begin{frame}{Vocabulario Importante}
\begin{itemize}
\item 🎯 Experimento: acción que hacemos
\end{itemize}
\begin{itemize}
\item 🎲 Resultado: lo que puede salir
\end{itemize}
\begin{itemize}
\item 📋 Espacio muestral: todos los resultados posibles
\end{itemize}
\begin{itemize}
\item ⭐ Evento: uno o más resultados que nos interesan
\end{itemize}
\begin{block}{Ejemplo}
Experimento: lanzar un dado. Espacio: {1,2,3,4,5,6}. Evento: sale par {2,4,6}
\end{block}
\end{frame}

\begin{frame}{Fórmula Básica de Probabilidad}
\begin{center}
\Large
$P(A) = \frac{\text{casos favorables}}{\text{casos totales}}$
\end{center}
\begin{itemize}
\item Casos favorables: resultados que queremos
\end{itemize}
\begin{itemize}
\item Casos totales: todos los resultados posibles
\end{itemize}
\begin{alertblock}{Nota}
¡Como contar tus opciones de ganar! 🎯
\end{alertblock}
\end{frame}

\begin{frame}{Ejemplo con un Dado 🎲}
\begin{exampleblock}{Problema}
Lanzamos un dado. ¿Probabilidad de sacar un número par?
\end{exampleblock}
\begin{itemize}
\item Resultados totales: 6 (del 1 al 6)
\end{itemize}
\begin{itemize}
\item Resultados favorables: 3 (el 2, 4 y 6)
\end{itemize}
\begin{align*}
P(par) = \frac{3}{6} = \frac{1}{2} = 0.5 = 50\%
\end{align*}
\begin{alertblock}{Nota}
¡Hay 50% de probabilidad! ¡La mitad de las veces!
\end{alertblock}
\end{frame}

\begin{frame}{Ejemplo con Cartas 🃏}
\begin{exampleblock}{Problema}
Sacas una carta de una baraja española (40 cartas)
\end{exampleblock}
\begin{itemize}
\item ¿Probabilidad de sacar un as?
\end{itemize}
\begin{itemize}
\item Ases totales: 4 (uno de cada palo)
\end{itemize}
\begin{itemize}
\item Cartas totales: 40
\end{itemize}
\begin{align*}
P(As) = \frac{4}{40} = \frac{1}{10} = 0.1 = 10\%
\end{align*}
\begin{itemize}
\item ¡Solo 10% de probabilidad de sacar un as!
\end{itemize}
\end{frame}

\begin{frame}{Regla del Complemento}
\begin{itemize}
\item El complemento es: lo que NO ocurre
\end{itemize}
\begin{center}
\Large
$P(\text{no A}) = 1 - P(A)$
\end{center}
\begin{alertblock}{Nota}
¡Si algo tiene 30% de pasar, tiene 70% de NO pasar!
\end{alertblock}
\begin{block}{Ejemplo}
Si P(lluvia) = 0.3, entonces P(no lluvia) = 1 - 0.3 = 0.7
\end{block}
\end{frame}

\begin{frame}{Ejemplo del Complemento}
\begin{exampleblock}{Problema}
En un dado: P(NO sacar 6) = ?
\end{exampleblock}
\begin{itemize}
\item Primero: P(sacar 6) = 1/6
\end{itemize}
\begin{align*}
P(NO sacar 6) = 1 - \frac{1}{6} = \frac{5}{6}
\end{align*}
\begin{itemize}
\item Tiene sentido: hay 5 números que no son 6
\end{itemize}
\begin{alertblock}{Nota}
¡El complemento es muy útil cuando es más fácil calcular lo opuesto! 🔄
\end{alertblock}
\end{frame}

\begin{frame}{Regla Aditiva - ¿A o B?}
\begin{itemize}
\item Para calcular: probabilidad de A O B
\end{itemize}
\begin{center}
\Large
$P(A \cup B) = P(A) + P(B) - P(A \cap B)$
\end{center}
\begin{itemize}
\item ¿Por qué restamos la intersección?
\end{itemize}
\begin{alertblock}{Nota}
¡Para no contar dos veces los casos comunes! 🔢
\end{alertblock}
\end{frame}

\begin{frame}{Eventos Mutuamente Excluyentes}
\begin{itemize}
\item Son eventos que NO pueden pasar al mismo tiempo
\end{itemize}
\begin{block}{Ejemplo}
En un dado: sacar 2 O sacar 5 (no pueden salir ambos)
\end{block}
\begin{itemize}
\item Si son mutuamente excluyentes:
\end{itemize}
\begin{center}
\Large
$P(A \cup B) = P(A) + P(B)$
\end{center}
\begin{alertblock}{Nota}
¡Solo sumamos porque no hay casos comunes! ➕
\end{alertblock}
\end{frame}

\begin{frame}{Ejemplo de Regla Aditiva Simple}
\begin{exampleblock}{Problema}
En una baraja: ¿probabilidad de As O Rey?
\end{exampleblock}
\begin{itemize}
\item Ases: 4, Reyes: 4, Total: 52 cartas
\end{itemize}
\begin{itemize}
\item Eventos mutuamente excluyentes (no puede ser ambos)
\end{itemize}
\begin{align*}
P(As) = \frac{4}{52}, \quad P(Rey) = \frac{4}{52}
\end{align*}
\begin{align*}
P(As \cup Rey) = \frac{4}{52} + \frac{4}{52} = \frac{8}{52} = \frac{2}{13}
\end{align*}
\end{frame}

\begin{frame}{Ejemplo con Intersección}
\begin{exampleblock}{Problema}
En 52 cartas: ¿probabilidad de Corazón O Figura?
\end{exampleblock}
\begin{itemize}
\item Corazones: 13, Figuras: 12 (J, Q, K de 4 palos)
\end{itemize}
\begin{itemize}
\item ¡PERO! 3 son ambos (J♥, Q♥, K♥)
\end{itemize}
\begin{align*}
P(♥) = \frac{13}{52}, \quad P(Figura) = \frac{12}{52}
\end{align*}
\begin{align*}
P(♥ \cup Figura) = \frac{13}{52} + \frac{12}{52} - \frac{3}{52} = \frac{22}{52}
\end{align*}
\end{frame}

\begin{frame}{Regla Multiplicativa - ¿A y B?}
\begin{itemize}
\item Para calcular: probabilidad de A Y B (ambos)
\end{itemize}
\begin{center}
\Large
$P(A \cap B) = P(A) \cdot P(B|A)$
\end{center}
\begin{itemize}
\item P(B|A) se lee: 'B dado que A ya ocurrió'
\end{itemize}
\begin{alertblock}{Nota}
¡Es como eventos en secuencia! 🔗
\end{alertblock}
\end{frame}

\begin{frame}{Eventos Independientes}
\begin{itemize}
\item Eventos donde uno NO afecta al otro
\end{itemize}
\begin{block}{Ejemplo}
Lanzar dos dados: el primero no afecta al segundo
\end{block}
\begin{itemize}
\item Si son independientes:
\end{itemize}
\begin{center}
\Large
$P(A \cap B) = P(A) \cdot P(B)$
\end{center}
\begin{alertblock}{Nota}
¡Solo multiplicamos! ✖️
\end{alertblock}
\end{frame}

\begin{frame}{Ejemplo de Eventos Independientes}
\begin{exampleblock}{Problema}
Lanzamos dos dados. ¿Probabilidad de que ambos sean 6?
\end{exampleblock}
\begin{itemize}
\item Primer dado: P(6) = 1/6
\end{itemize}
\begin{itemize}
\item Segundo dado: P(6) = 1/6
\end{itemize}
\begin{itemize}
\item Son independientes (uno no afecta al otro)
\end{itemize}
\begin{align*}
P(6 \cap 6) = \frac{1}{6} \times \frac{1}{6} = \frac{1}{36}
\end{align*}
\begin{alertblock}{Nota}
¡Muy difícil! Solo 2.78% de probabilidad 🎲🎲
\end{alertblock}
\end{frame}

\begin{frame}{Ejemplo de Eventos Dependientes}
\begin{exampleblock}{Problema}
Urna con 5 rojas y 3 azules. Sacas 2 bolas SIN reposición
\end{exampleblock}
\begin{itemize}
\item ¿Probabilidad de 2 rojas?
\end{itemize}
\begin{itemize}
\item Primera bola roja: P(R₁) = 5/8
\end{itemize}
\begin{itemize}
\item Segunda bola roja (ya sacaste una): P(R₂|R₁) = 4/7
\end{itemize}
\begin{align*}
P(R₁ \cap R₂) = \frac{5}{8} \times \frac{4}{7} = \frac{20}{56} = \frac{5}{14}
\end{align*}
\end{frame}

\begin{frame}{Con Reposición vs Sin Reposición}
\begin{center}
\begin{tabular}{|c|c|)}
\toprule
Con reposición & Sin reposición \\
\midrule
Devuelves la bola & No devuelves \\
Total no cambia & Total disminuye \\
Eventos independientes & Eventos dependientes \\
\bottomrule
\end{tabular}
\end{center}
\begin{alertblock}{Nota}
¡La reposición es clave para determinar independencia! 🔑
\end{alertblock}
\end{frame}

\begin{frame}{Problema de la Moneda 🪙}
\begin{exampleblock}{Problema}
Lanzas una moneda 3 veces. ¿Probabilidad de 3 caras?
\end{exampleblock}
\begin{itemize}
\item Cada lanzamiento es independiente
\end{itemize}
\begin{itemize}
\item P(cara) = 1/2 en cada lanzamiento
\end{itemize}
\begin{align*}
P(CCC) = \frac{1}{2} \times \frac{1}{2} \times \frac{1}{2} = \frac{1}{8}
\end{align*}
\begin{alertblock}{Nota}
¡Solo 12.5% de probabilidad! 🎯
\end{alertblock}
\end{frame}

\begin{frame}{Problema Combinado}
\begin{exampleblock}{Problema}
Urna: 5 bolas rojas, 3 azules. Un sorteo:
\end{exampleblock}
\begin{itemize}
\item a) ¿P(roja)?
\end{itemize}
\begin{itemize}
\item b) ¿P(roja O azul)?
\end{itemize}
\begin{itemize}
\item c) ¿P(2 rojas seguidas SIN reposición)?
\end{itemize}
\begin{alertblock}{Nota}
¡Vamos a resolverlo paso a paso! 🔍
\end{alertblock}
\end{frame}

\begin{frame}{Solución Parte a y b}
\begin{itemize}
\item Total de bolas: 5 + 3 = 8
\end{itemize}
\begin{align*}
a) \, P(R) = \frac{5}{8}
\end{align*}
\begin{itemize}
\item Roja O azul = todas las bolas (mutuamente excluyentes)
\end{itemize}
\begin{align*}
b) \, P(R \cup A) = \frac{5}{8} + \frac{3}{8} = \frac{8}{8} = 1
\end{align*}
\begin{alertblock}{Nota}
¡Seguro sacarás roja o azul porque no hay otras! 100%
\end{alertblock}
\end{frame}

\begin{frame}{Solución Parte c}
\begin{itemize}
\item Primera roja: P(R₁) = 5/8
\end{itemize}
\begin{itemize}
\item Segunda roja SIN reposición: P(R₂|R₁) = 4/7
\end{itemize}
\begin{itemize}
\item (quedan 4 rojas de 7 bolas totales)
\end{itemize}
\begin{align*}
c) \, P(R₁ \cap R₂) = \frac{5}{8} \times \frac{4}{7} = \frac{20}{56} = \frac{5}{14}
\end{align*}
\begin{alertblock}{Nota}
¡Aproximadamente 35.7% de probabilidad! 📊
\end{alertblock}
\end{frame}

\begin{frame}{Problema de Cartas Avanzado 🎴}
\begin{exampleblock}{Problema}
Baraja de 52 cartas. Sacas 2 cartas sin reposición:
\end{exampleblock}
\begin{itemize}
\item ¿Probabilidad de que ambas sean Ases?
\end{itemize}
\begin{itemize}
\item Primera carta: P(As) = 4/52
\end{itemize}
\begin{itemize}
\item Segunda carta: P(As|As) = 3/51
\end{itemize}
\begin{align*}
P(As \cap As) = \frac{4}{52} \times \frac{3}{51} = \frac{12}{2652} = \frac{1}{221}
\end{align*}
\begin{itemize}
\item ¡Menos de 0.5%! Muy difícil 🃏
\end{itemize}
\end{frame}

\begin{frame}{Problema de Aplicación Real}
\begin{exampleblock}{Problema}
En una clase: 60% estudia, 40% no estudia
\end{exampleblock}
\begin{itemize}
\item De los que estudian: 90% aprueba
\end{itemize}
\begin{itemize}
\item De los que no estudian: 30% aprueba
\end{itemize}
\begin{itemize}
\item ¿Probabilidad de que un estudiante al azar apruebe?
\end{itemize}
\begin{alertblock}{Nota}
¡Vamos a usar todo lo aprendido! 💪
\end{alertblock}
\end{frame}

\begin{frame}{Solución del Problema Real}
\begin{itemize}
\item P(aprueba) = P(estudia Y aprueba) O P(no estudia Y aprueba)
\end{itemize}
\begin{align*}
P(E \cap A) = 0.6 \times 0.9 = 0.54
\end{align*}
\begin{align*}
P(NE \cap A) = 0.4 \times 0.3 = 0.12
\end{align*}
\begin{align*}
P(A) = 0.54 + 0.12 = 0.66 = 66\%
\end{align*}
\begin{alertblock}{Nota}
¡66% de estudiantes aprueban! Pero estudiar aumenta mucho tus chances 📚
\end{alertblock}
\end{frame}

\begin{frame}{Tips para Resolver Problemas}
\begin{itemize}
\item 1️⃣ Identifica el experimento y espacio muestral
\end{itemize}
\begin{itemize}
\item 2️⃣ ¿Son eventos independientes o dependientes?
\end{itemize}
\begin{itemize}
\item 3️⃣ ¿Hay reposición o no?
\end{itemize}
\begin{itemize}
\item 4️⃣ ¿Es 'O' (suma) o 'Y' (multiplicación)?
\end{itemize}
\begin{itemize}
\item 5️⃣ Verifica que tu respuesta esté entre 0 y 1
\end{itemize}
\begin{alertblock}{Nota}
¡La práctica hace al maestro! 🌟
\end{alertblock}
\end{frame}

\begin{frame}{Ejercicio Final de Práctica}
\begin{exampleblock}{Problema}
En una bolsa: 4 bolas verdes, 3 amarillas, 2 negras
\end{exampleblock}
\begin{itemize}
\item a) ¿P(verde)?
\end{itemize}
\begin{itemize}
\item b) ¿P(NO negra)?
\end{itemize}
\begin{itemize}
\item c) ¿P(verde O amarilla)?
\end{itemize}
\begin{itemize}
\item d) ¿P(2 verdes seguidas SIN reposición)?
\end{itemize}
\begin{alertblock}{Nota}
¡Intenta resolverlo antes de ver la solución!
\end{alertblock}
\end{frame}

\begin{frame}{Solución del Ejercicio Final}
\begin{itemize}
\item Total: 4 + 3 + 2 = 9 bolas
\end{itemize}
\begin{align*}
a) \, P(V) = \frac{4}{9}
\end{align*}
\begin{align*}
b) \, P(\neg N) = 1 - \frac{2}{9} = \frac{7}{9}
\end{align*}
\begin{align*}
c) \, P(V \cup Am) = \frac{4}{9} + \frac{3}{9} = \frac{7}{9}
\end{align*}
\begin{align*}
d) \, P(V₁ \cap V₂) = \frac{4}{9} \times \frac{3}{8} = \frac{12}{72} = \frac{1}{6}
\end{align*}
\end{frame}

\begin{frame}{¡Resumen de la Clase!}
\begin{itemize}
\item ✅ Probabilidad básica: casos favorables / totales
\end{itemize}
\begin{itemize}
\item ✅ Complemento: P(no A) = 1 - P(A)
\end{itemize}
\begin{itemize}
\item ✅ Regla aditiva (O): sumar probabilidades
\end{itemize}
\begin{itemize}
\item ✅ Regla multiplicativa (Y): multiplicar probabilidades
\end{itemize}
\begin{itemize}
\item ✅ Independencia vs dependencia
\end{itemize}
\begin{itemize}
\item ✅ Con/sin reposición
\end{itemize}
\begin{alertblock}{Nota}
¡Ahora puedes calcular probabilidades como un experto! 🎲🎯
\end{alertblock}
\end{frame}


% Diapositiva final
\begin{frame}
\begin{center}
\Huge ¿Preguntas?
\end{center}
\end{frame}

\end{document}